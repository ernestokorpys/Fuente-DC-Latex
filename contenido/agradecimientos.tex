% LOS RECONOCIMIENTOS
% -------------------

% Aquí se escribe la dedicatoria del proyecto y los agradecimientos. El entorno 'reconocimiento' tiene la estructura \begin{reconocimiento}{Dedicatoria} Agradecimientos \end{reconocimiento}

\begin{reconocimiento}{Dedicado a nuestras familias y amigos Saludos.}

Korpys Ernesto.
Agradezco de corazón a mi familia por su inquebrantable apoyo a lo largo de toda mi vida. En especial, a mi madre Gladys, cuyo amor y sacrificio han sido mi mayor inspiración y motor para alcanzar mis metas.
Agradezco enormemente a mi compañero de proyecto, Fernando, quien no solo fue mi compañero de trabajo, sino también un amigo invaluable durante esta travesía académica. Su colaboración y compañerismo fueron fundamentales para el éxito de este proyecto.
A mis amigos presentes, les doy las gracias por su constante ánimo y respaldo, por compartir conmigo momentos de alegría y por ser un pilar fundamental en mi vida.
Expreso mi profundo agradecimiento al equipo docente, a los ingenieros Botteron, Fernandez y Kolodziej, quienes no solo compartieron su conocimiento y experiencia conmigo, sino que también me brindaron su apoyo académico cuando más lo necesité. Gracias por ser guías en este viaje de aprendizaje y crecimiento profesional. \par 
\par 

\textbf{Krindges Fernando:} \par 
Llegar hasta aquí ha sido un camino de esfuerzo, dedicación y, sobre todo, de apoyo de quienes creyeron en mí. A mi madre, Vogel Claudia, debo el mayor de mis agradecimientos. Su apoyo incondicional, sus sacrificios y su fortaleza fueron el pilar que me impulsó a no rendirme. Gracias, mamá, por ser mi inspiración y por brindarme todas las oportunidades para que pudiera alcanzar mis metas.\par 

También agradezco de corazón a Korpys Ernesto, mi amigo y compañero de proyecto, con quien enfrenté desafíos, noches de estudio y momentos inolvidables. Ernesto, tu dedicación y amistad hicieron este recorrido mucho más enriquecedor. Gracias por tu esfuerzo y por ser un apoyo constante en esta etapa de nuestras vidas.\par 

Finalmente, quiero expresar mi gratitud hacia los ingenieros que fueron una guía inestimable, especialmente a Fernández Guillermo y Botteron Fernando. Con paciencia y generosidad, compartieron sus conocimientos y nos guiaron, haciendo de cada lección una puerta abierta a nuevas oportunidades. Ellos, junto con todos los docentes que nos acompañaron, hicieron de la ingeniería no solo una carrera, sino una vocación apasionante.\par 
\end{reconocimiento}