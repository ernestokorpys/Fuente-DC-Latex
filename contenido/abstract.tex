% EL RESUMEN EN INGLÉS
% --------------------

\begin{theabstract} {Direct Current Power Supply:\\Digital Redesign of an Analog Control for an adjustable Power Supply} {Voltage source, Control loop, Digital Control, Current Control, Voltage Control}

The project focuses on the modernization of an existing power supply, predominantly analog, which offers a variable voltage range from 0V to 30V and an adjustable current from 0A to 3A. This will be achieved through the implementation of digital control, allowing precise regulation of the output voltage and current, as well as the establishment of digital control loops to ensure output stability under various load conditions.\par

The essence of this project lies in the use of a digital signal controller to precisely adjust the desired voltage and current at the output of the power supply. This is made possible through the implemented digital control strategy and the capability for the user to input the desired voltage and current values using an alphanumeric keypad and rotary encoder potentiometers. Additionally, an integrated display provides real-time visual feedback, showing both the set values and the actual output values. \par

At the core of the control system, an Arduino NANO programmable device is used, notable for its ability to efficiently manage all the operations required by the power supply. A key feature of this power supply is that it includes galvanic isolation in the signal acquisition stage entering the microcontroller and between the digital controller and the power stage. \par

\section*{Objectives}

\begin{itemize}
    \item Power supply providing a variable output voltage from 0V to 30V and an adjustable current from 0A to 3A.
    \item Implement digital control loops to ensure the regulation of output voltage and current, maintaining stability under various load conditions.
    \item Allow easy and precise configuration of output voltage and current using an input system, such as a rotary encoder or a keypad.
    \item Integrate a display showing both user-set values and actual output values, providing real-time visual feedback.
    \item Ensure complete galvanic isolation between the power stage and digital control, ensuring system safety and reliability.
    \item Wireless connectivity is not required.
\end{itemize}

\end{theabstract}

% El entorno 'theabstract' tiene el formato \begin{theabstract}{A} ...B... \end{theabstract} donde A es el título del proyecto traducido de inglés a español y B es el contenido, en inglés, del resumen. Se recomienda buscar ayuda calificada para la elaboración y/o revisión de este resumen.