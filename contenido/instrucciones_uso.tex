\chapter{Instrucciones de uso.}
% ----------------------

\label{C:Forma de operar la fuente DC}

\section{Ciclo de funcionamiento}

Esta sección del informe detalla el procedimiento adecuado para la operación del equipo bajo condiciones normales. Se describen los pasos necesarios desde la energización de los transformadores hasta la configuración de los valores, con el fin de garantizar un funcionamiento eficiente y seguro del equipo.

\begin{enumerate}
    \item Energización de los transformadores.
    \item Carga de la pantalla de inicio del menú.
    \item Tecla B para moverse sobre el menú.
    \item Selector de modo. Tensión. Corriente. Rampa.
    \item Tecla C para confirmar el modo escogido.
    \item Tecla A para entrar al modo de edición de valores.
    \item Cargar Valores con teclas numéricas.
    \item Tecla \# para poner en marcha la fuente desde la pantalla del modo.
    \item Tecla * para conectar la carga.
    \item Desactivar la fuente en el menú con tecla D.
\end{enumerate}

\subsection{Diagrama de control de funcionamiento.}

El diagrama de control del funcionamiento es una representación visual esencial que ilustra de manera clara y concisa los pasos y procesos involucrados en el manejo adecuado de un sistema o equipo. Este tipo de diagrama proporciona una guía visual detallada que facilita la comprensión y la ejecución de las tareas necesarias para operar el equipo de manera eficiente. El que se encuentra a continuación resume lo desarrollado en la parte superior de cual sería un ciclo normal de utilización sin entrar en detalle acerca de errores y fallas imprevistas. 

\begin{figure}[H]
    \centering
    \includegraphics[scale=0.7]{./imagenes/funcionamiento_normal.jpg}
    \caption{Diagrama de bloques del funcioanmiento estandar.}
    \label{F:funcionamiento_normal}
\end{figure}

\section{Uso de teclado} 
El teclado es una interfaz fundamental en el proceso de interacción del usuario con la fuente de alimentación, permitiendo la configuración de parámetros y el control de diferentes modos de operación. Cada tecla tiene asignada una función específica para facilitar la navegación y la manipulación de la configuración. A continuación, se presenta un resumen de la función que realiza cada tecla:
\begin{itemize}
    \item A: Editar valores.
    \item B: Moverse sobre el menú.
    \item C: Confirmar.
    \item D: Volver atrás.
    \item *: Conectar la carga.
    \item \#: Poner en marcha el modo.
\end{itemize}

\section{Pantallas Disponibles} 
En el display se encuentran disponibles seis pantallas distintas, cada una con un propósito específico para la interacción del usuario con la fuente. A continuación, se detalla el contenido y la función de cada una.
Esta estructura proporciona al usuario una interfaz intuitiva y clara para interactuar con la fuente, facilitando la configuración y el monitoreo de los parámetros de salida.
\begin{enumerate}
    \item \textbf{Pantalla Principal - Menú Principal}: Esta pantalla representa el menú principal desde el cual se puede navegar para configurar la fuente. Mediante un puntero, el usuario puede seleccionar el modo de operación deseado.
    \item \textbf{Pantalla de Modo Tensión}: Al seleccionar este modo, la pantalla mostrará los valores configurados para la tensión. Además, proporcionará una visualización en tiempo real de las magnitudes de tensión registradas.
    \item \textbf{Pantalla de Modo Corriente}: Similar al modo de tensión, esta pantalla muestra los valores configurados para la corriente. No incluye un campo para la tensión deseada, ya que se activa exclusivamente el modo de corriente.
    \item \textbf{Pantalla de Modo Rampa}: Aquí se visualizan los valores de tiempo y tensión configurados, junto con los valores registrados en tiempo real y el tiempo transcurrido desde el inicio del modo de rampa.
    \item \textbf{Pantalla de Carga Inválida}: Esta pantalla muestra un mensaje de error cuando los parámetros cargados no son válidos dentro de los límites constructivos de la fuente.
    \item \textbf{Pantalla de Carga de Valores}: En esta pantalla se muestran los valores deseados de los parámetros. Se pueden cargar uno a la vez utilizando el teclado alfanumérico.
\end{enumerate}













