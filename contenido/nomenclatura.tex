% LA NOMENCLATURA
% ---------------

% La nomenclatura se realiza con el paquete 'nomencl'. Para ingresar un nuevo elemento, se debe usar el comando \nomenclature{símbolo}{definición}, ya sea en este archivo nomenclatura.tex (más fácil para encontrar y editar), o en cualquier parte del documento (probablemente cuando se introduce una nueva variable o constante). Para más opciones del paquete, favor referirse a su documentación (https://www.ctan.org/pkg/nomencl). También hay una buena guía de uso en https://www.sharelatex.com/learn/Nomenclatures.

% Formato recomendado
% -------------------

% Variable o constante matemática
% \nomenclature{$V$}{Tensión eléctrica}

% Acrónimo
% \nomenclature{TBH}{Para ser honesto (del inglés \textit{To Be Honest})}

% Si únicamente existen acrónimos del inglés, se puede omitir la frase 'del inglés'. La definición no tiene punto al final.

\nomenclature{$R$}{Resistencia eléctrica}
\nomenclature{$I$}{Corriente eléctrica}
\nomenclature{$V$}{Tensión eléctrica}
\nomenclature{IEEE}{Instituto de Ingenieros Eléctricos y Electrónicos (del inglés \textit{Institute of Electrical and Electronics Engineers})}
\nomenclature{FIO}{Facultad de Ingeniería}
\nomenclature{UNaM}{Universidad Nacional de Misiones}
\nomenclature{ROM}{Read Only Memory}
\nomenclature{RAM}{Random Access Memory}
\nomenclature{CMOS}{Complementary Metal-Oxide Semiconductor}
\nomenclature{E/S}{Entrada/Salida}
\nomenclature{TTL}{Transistor-Transistor Logic}
\nomenclature{I\textsuperscript{2}C}{Inter-Integrated Circuit}
\nomenclature{UART}{Universal Asynchronous Receiver-Transmitter}
\nomenclature{SPI}{Serial Peripheral Interface}
\nomenclature{COSMAC}{Complementary Symmetry Monolithic Array Computer}
\nomenclature{MSB}{Most Significant Bit}
\nomenclature{LSB}{Least Sifnificant Bit}
\nomenclature{EPROM}{Erasable Programmable Read-Only Memory}
\nomenclature{EEPROM}{Electrically Erasable Programmable Read-Only Memory}
\nomenclature{RCA}{Radio Corporation of America}
\nomenclature{EDA}{Electronic Design Automation}
\nomenclature{$VEEEEE$}{asdasdasdas eléctrica}
\nomenclature{PCB}{Circuito impreso (del inglés \textit{Printed Circuit Board})}
\nomenclature{PAL}{Programmable Array Logic}
\nomenclature{GAL}{Generic Array Logic}
\nomenclature{PLC}{Programmable Logic Controller}
\nomenclature{KiCad}{Software para el diseño de esquemáticos y PCBs de circuitos electrónicos}
\nomenclature{MRD}{Memory Read}
\nomenclature{MWR}{Memory Read}

\printnomenclature