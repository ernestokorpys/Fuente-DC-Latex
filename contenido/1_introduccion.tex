% ----------------------
  \chapter{Introducción}
% ----------------------

\label{C:introduccion}

El proyecto se centra en la modernización de una fuente de alimentación preexistente, en su mayoría analógica, de una tensión variable desde 0V hasta 30V y una corriente ajustable desde 0A hasta 3A, mediante la implementación de un control digital para la regulación precisa de la tensión y corriente de salida, así como la implementación de lazos de control digital para garantizar la estabilidad de la salida en diversas condiciones de carga. \par
La esencia de este proyecto radica en la utilización de un controlador digital de señales para ajustar la salida de la fuente de alimentación, introduciendo el valor deseado a través de un teclado o un \textit{encoder} rotativo. Este enfoque proporciona al usuario la capacidad de configurar fácilmente los valores deseados de tensión y corriente de salida, mientras que un display integrado ofrece una retroalimentación visual en tiempo real, mostrando tanto los valores establecidos como los valores reales de salida.\par
Como núcleo de control, se emplea el microcontrolador digital \entreComillas{Arduino NANO}, destacando su bajo costo, accesibilidad y la disponibilidad de librerías creadas por la gran comunidad que lo respalda. Resulta importante mencionar que este diseño no requiere de conexión inalámbrica, lo que simplifica su implementación y uso.


%%%%%%%%%%%%%%%%%%%%%%%%%%%%%%%%%%%%%%%%%%%%%