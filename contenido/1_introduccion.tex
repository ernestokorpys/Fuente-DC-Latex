% ----------------------
  \chapter{Introducción}
% ----------------------

\label{C:introduccion} 

El proyecto se enfoca en la modernización de una fuente de alimentación existente, predominantemente analógica, que ofrece un rango de tensión variable de 0V a 30V y una corriente ajustable de 0A a 3A. Esto se logrará mediante la implementación de un control digital que permitirá una regulación precisa de la tensión y la corriente de salida, así como el establecimiento de lazos de control digital que asegurarán la estabilidad de la salida bajo diversas condiciones de carga. \par

La esencia de este proyecto radica en la utilización de un controlador digital de señales para ajustar la salida de la fuente de alimentación, introduciendo el valor deseado a través de un teclado o un \textit{encoder} rotativo. Este enfoque proporciona al usuario la capacidad de configurar fácilmente los valores deseados de tensión y corriente de salida, mientras que un display integrado ofrece una retroalimentación visual en tiempo real, mostrando tanto los valores establecidos como los valores reales de salida.\par

Como núcleo de control, se emplea el microcontrolador digital \entreComillas{Arduino NANO}, destacando su bajo costo, accesibilidad y la disponibilidad de librerías creadas por la gran comunidad que lo respalda. Resulta importante mencionar que este diseño no requiere de conexión inalámbrica, lo que simplifica su implementación y uso.


%%%%%%%%%%%%%%%%%%%%%%%%%%%%%%%%%%%%%%%%%%%%%