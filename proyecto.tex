% PLANTILLA Y GUÍA DEL TRABAJO ESCRITO
% ------------------------------------

\immediate\write18{makeindex -s nomencl.ist -o "proyecto.nls" "proyecto.nlo"}

% Tipo de documento
\documentclass[final]{proyectoelectronico}

% A. PAQUETES Y MACROS ESPECIALES -------
% Paquetes y definiciones que no están incluidos 
% en la clase proyectoelectrico.cls o que son 
% propios del proyecto.
\input{configuracion.tex}

% Tipografía
\usepackage{libertine}
\usepackage{libertinust1math}
\usepackage[T1]{fontenc}
\renewcommand*{\ttdefault}{cmtt}
\newcommand{\entreComillas}[1]{“#1”} % para que esté entre comillas
\newcommand{\listaVacio}{Varios} % celda vacia de la lista de componentes 

%---------------------------------------

% B. DATOS ------------------------------
% Todos los nombres incluyen dos apellidos,
% acentos y signos de puntuación apropiados.
% Revisar las recomendaciones sobre el título
% y los nombres de los profesores en la guía.

% Título del proyecto
\titulo{Digitalización de Control Analógico para Fuente de Alimentación Ajustable}

% Autor (nombre y carné)
\autoruno{Korpys Ernesto Andrés}
\carneuno{K-3290/5} % legajo
\emailuno{ernesto.korpys@gmail.com}

\autordos{Fernando Natanael Krindgester}
\carnedos{MODIFICAR Legajo del autor 2} % legajo
\emaildos{krindgesfer@gmail.com}

%\autortres{Nombre del tercer integrante}
%\carnetres{Legajo del tercer integrante} % legajo
%\emailtres{Email del tercer integrante}

% Tutores (nombre y correo)

\tutoruno{Botterón Fernando}
\tutemailuno{botteron@fio.unam.edu.ar}

\tutordos{Maxit Alejandro Germán}
\tutemaildos{alejandro.maxit@fio.unam.edu.ar}

% Profesor(a) guía
\guia{Ricardo Andrés Korpys}

% Profesores lectores 
\lectorA{Guillermo Alfredo Fernandez}
\lectorB{Alejandro Germán Maxit}

% Fecha de entrega del trabajo escrito
\mes{11}	% Número del mes
\ano{2024}	% Formato AAAA
% ---------------------------------------

% C. CONTENIDOS -------------------------

%\hypersetup{citecolor=black} % esto es el color en el que aparecen las citas cuando se usa el comando \cite{}, por defecto las llaves están el negro y el número en azul

%%%%%%%%%%%%%%%%%%%
\begin{document}
%%%%%%%%%%%%%%%%%%%

\frontmatter

% 1. PORTADA
\portada

% 2. HOJA DE APROBACIÓN
\iffinal
\aprobacion
\fi

% 3. RESUMEN (EN ESPAÑOL E INGLÉS)
% EL RESUMEN
% ----------

\begin{resumen}{Fuente de tensión, Lazo de control, Control Digital, Control de Corriente, Control de Tensión.}

El proyecto se centra en la modernización de una fuente de alimentación preexistente, en su mayoría analógica, de una tensión variable desde 0V hasta 30V y una corriente ajustable desde 0A hasta 3A, mediante la implementación de un control digital para la regulación precisa de la tensión y corriente de salida, así como la implementación de lazos de control digital para garantizar la estabilidad de la salida en diversas condiciones de carga. \par

La esencia de este proyecto radica en la utilización de un controlador digital de señales para poder ajustar con precisión la tensión y la corriente deseadas en la salida de la fuente de alimentación. Esto es posible gracias a la estrategia de control digital implementada y a la capacidad incorporada de que el usuario pueda incorporar los valores deseados de tensión y corriente mediante un teclado alfanumérico y también, mediante codificadores potenciómetros rotativos. A su vez, un display integrado ofrece una retroalimentación visual en tiempo real, mostrando tanto los valores establecidos como los valores reales de salida.\par

Como núcleo del control, se emplea un dispositivo programable Arduino NANO, del cual se destaca su capacidad para gestionar eficientemente todas las operaciones requeridas por la fuente de alimentación. Una característica esencial de esta fuente de alimentación es que presenta aislación galvánica en la etapa de adquisición de las señales que ingresan al microcontrolador y entre el controlador digital y la etapa de potencia.\par

\section*{Objetivos}

\begin{itemize}
    \item Fuente de alimentación que proporcione una tensión de salida variable desde 0V hasta 30V y una corriente ajustable desde 0A hasta 3A.
    \item Implementar la funcionalidad de lazos de control digital para garantizar una regulación de la tensión y corriente de salida, asegurando estabilidad en diversas condiciones de carga.
    \item Permitir la configuración fácil y precisa de la tensión y corriente de salida mediante un sistema de entrada, como un encoder rotativo o un teclado.
    \item Integrar un display que presente tanto los valores establecidos por el usuario como los valores reales de salida, proporcionando retroalimentación visual en tiempo real.
    \item Garantizar un aislamiento galvánico completo entre la etapa de potencia y el control digital, asegurando la seguridad y fiabilidad del sistema.
    \item No es necesario disponer de conexión inalámbrica.
\end{itemize}

\end{resumen}

% EL RESUMEN EN INGLÉS
% --------------------

\begin{theabstract} {Direct Current Power Supply:\\Digital Redesign of an Analog Control for an adjustable Power Supply} {Voltage source, Control loop, Digital, Current Control, Voltage Control}

The project focuses on modernizing an existing power supply, predominantly analog, with a variable voltage range from 0V to 30V and an adjustable current range from 0A to 3A. This is achieved through the implementation of digital control for precise regulation of both voltage and current output, alongside the incorporation of digital control loops to ensure output stability under various load conditions.

The essence of this project lies in the utilization of a digital signal controller to adjust the power supply's output via a keypad. This approach grants users the ability to easily configure desired voltage and current values, while an integrated display provides real-time visual feedback, showcasing both set and actual output values.

At the core of control, the project employs the Arduino NANO, notable for its efficient management of system operations. It is pertinent to mention that this design does not necessitate wireless connectivity, simplifying its implementation and usage.

\section*{Objectives}

\begin{itemize}
    \item Power supply providing a variable output voltage from 0V to 30V and an adjustable current from 0A to 3A.
    \item Implement digital control loops to ensure the regulation of output voltage and current, maintaining stability under various load conditions.
    \item Allow easy and precise configuration of output voltage and current using an input system, such as a rotary encoder or a keypad.
    \item Integrate a display showing both user-set values and actual output values, providing real-time visual feedback.
    \item Ensure complete galvanic isolation between the power stage and digital control, ensuring system safety and reliability.
    \item Wireless connectivity is not required.
\end{itemize}

\end{theabstract}

% El entorno 'theabstract' tiene el formato \begin{theabstract}{A} ...B... \end{theabstract} donde A es el título del proyecto traducido de inglés a español y B es el contenido, en inglés, del resumen. Se recomienda buscar ayuda calificada para la elaboración y/o revisión de este resumen.

% 4. RECONOCIMIENTOS
\iffinal
% LOS RECONOCIMIENTOS
% -------------------

% Aquí se escribe la dedicatoria del proyecto y los agradecimientos. El entorno 'reconocimiento' tiene la estructura \begin{reconocimiento}{Dedicatoria} Agradecimientos \end{reconocimiento}

\begin{reconocimiento}{Dedicado a nuestras familias y amigos Saludos.}

Korpys Ernesto.
Agradezco de corazón a mi familia por su inquebrantable apoyo a lo largo de toda mi vida. En especial, a mi madre Gladys, cuyo amor y sacrificio han sido mi mayor inspiración y motor para alcanzar mis metas.
Agradezco enormemente a mi compañero de proyecto, Fernando, quien no solo fue mi compañero de trabajo, sino también un amigo invaluable durante esta travesía académica. Su colaboración y compañerismo fueron fundamentales para el éxito de este proyecto.
A mis amigos presentes, les doy las gracias por su constante ánimo y respaldo, por compartir conmigo momentos de alegría y por ser un pilar fundamental en mi vida.
Expreso mi profundo agradecimiento al equipo docente, a los ingenieros Botteron, Fernandez y Kolodziej, quienes no solo compartieron su conocimiento y experiencia conmigo, sino que también me brindaron su apoyo académico cuando más lo necesité. Gracias por ser guías en este viaje de aprendizaje y crecimiento profesional.


\end{reconocimiento}
\fi

% 5. TABLAS DE CONTENIDO, FIGURAS Y TABLAS
\tableofcontents
\listoffigures
\listoffotos
\listoftables
%\lstlistoflistings

% 6. NOMENCLATURA
%\input{nomenclature.tex}
\nomenclature{$R$}{Resistencia eléctrica}
% LA NOMENCLATURA
% ---------------

% La nomenclatura se realiza con el paquete 'nomencl'. Para ingresar un nuevo elemento, se debe usar el comando \nomenclature{símbolo}{definición}, ya sea en este archivo nomenclatura.tex (más fácil para encontrar y editar), o en cualquier parte del documento (probablemente cuando se introduce una nueva variable o constante). Para más opciones del paquete, favor referirse a su documentación (https://www.ctan.org/pkg/nomencl). También hay una buena guía de uso en https://www.sharelatex.com/learn/Nomenclatures.

% Formato recomendado
% -------------------

% Variable o constante matemática
% \nomenclature{$V$}{Tensión eléctrica}

% Acrónimo
% \nomenclature{TBH}{Para ser honesto (del inglés \textit{To Be Honest})}

% Si únicamente existen acrónimos del inglés, se puede omitir la frase 'del inglés'. La definición no tiene punto al final.

\nomenclature{$R$}{Resistencia eléctrica}
\nomenclature{$I$}{Corriente eléctrica}
\nomenclature{$V$}{Tensión eléctrica}
\nomenclature{IEEE}{Instituto de Ingenieros Eléctricos y Electrónicos (del inglés \textit{Institute of Electrical and Electronics Engineers})}
\nomenclature{FIO}{Facultad de Ingeniería}
\nomenclature{UNaM}{Universidad Nacional de Misiones}
\nomenclature{ROM}{Read Only Memory}
\nomenclature{RAM}{Random Access Memory}
\nomenclature{CMOS}{Complementary Metal-Oxide Semiconductor}
\nomenclature{E/S}{Entrada/Salida}
\nomenclature{TTL}{Transistor-Transistor Logic}
\nomenclature{I\textsuperscript{2}C}{Inter-Integrated Circuit}
\nomenclature{UART}{Universal Asynchronous Receiver-Transmitter}
\nomenclature{SPI}{Serial Peripheral Interface}
\nomenclature{COSMAC}{Complementary Symmetry Monolithic Array Computer}
\nomenclature{MSB}{Most Significant Bit}
\nomenclature{LSB}{Least Sifnificant Bit}
\nomenclature{EPROM}{Erasable Programmable Read-Only Memory}
\nomenclature{EEPROM}{Electrically Erasable Programmable Read-Only Memory}
\nomenclature{RCA}{Radio Corporation of America}
\nomenclature{EDA}{Electronic Design Automation}
\nomenclature{$VEEEEE$}{asdasdasdas eléctrica}
\nomenclature{PCB}{Circuito impreso (del inglés \textit{Printed Circuit Board})}
\nomenclature{PAL}{Programmable Array Logic}
\nomenclature{GAL}{Generic Array Logic}
\nomenclature{PLC}{Programmable Logic Controller}
\nomenclature{KiCad}{Software para el diseño de esquemáticos y PCBs de circuitos electrónicos}
\nomenclature{MRD}{Memory Read}
\nomenclature{MWR}{Memory Read}

\printnomenclature
\mainmatter

% 7. CAPÍTULOS
% ----------------------
  \chapter{Introducción}
% ----------------------

\label{C:introduccion}

El proyecto se centra en la modernización de una fuente de alimentación preexistente, en su mayoría analógica, de una tensión variable desde 0V hasta 30V y una corriente ajustable desde 0A hasta 3A, mediante la implementación de un control digital para la regulación precisa de la tensión y corriente de salida, así como la implementación de lazos de control digital para garantizar la estabilidad de la salida en diversas condiciones de carga. 
La esencia de este proyecto radica en la utilización de un controlador digital de señales para ajustar la salida de la fuente de alimentación, a través de un teclado. Este enfoque proporciona al usuario la capacidad de configurar fácilmente los valores deseados de tensión y corriente de salida, mientras que un display integrado ofrece una retroalimentación visual en tiempo real, mostrando tanto los valores establecidos como los valores reales de salida.
Como núcleo de control, se emplea el controlador digital de señales dsPIC30F4011 (40-Pin PDIP), destacando su capacidad para gestionar eficientemente las operaciones del sistema. Es importante mencionar que este diseño no requiere de conexión inalámbrica, lo que simplifica su implementación y uso.
 \cite{plantilla_universidad_de_costa_rica}.


%%%%%%%%%%%%%%%%%%%%%%%%%%%%%%%%%%%%%%%%%%%%%%


\input{contenido/6_IntroduccionTeorica}
\chapter{Modificación de la Fuente DC anterior}
% ----------------------

\label{C:Sobre la fuente anterior}

\section{Sobre la fuentes de alimentación anterior}
La revisión y adaptación del trabajo previo titulado "Diseño y construcción de una fuente de alimentación DC lineal con control digital de tensión y corriente" llevado a cabo por Eduardo Javier Matijak y Joaquín Pelinski, documentado en su publicación [23], sirve como punto de partida para comprender las mejoras implementadas en la fuente de alimentación DC que se examina en este informe. \textbf{Invitamos cordialmente al lector interesado a consultar dicho trabajo para obtener 
una comprensión más completa de los fundamentos sobre los cuales se basa este análisis.}
Este documento se centra en analizar y discutir las modificaciones realizadas en la fuente de alimentación, específicamente la transición de su mayoría analógica a una configuración digital. Entre los principales cambios introducidos se destacan los siguientes aspectos:

\subsection{Circuito Fijador de Referencia para los Transistores}
El circuito fijador de referencia para los transistores ha sido modificado para incorporar la salida de un Convertidor Analógico-Digital (DAC). El DAC es ahora responsable de aplicar niveles de voltaje acorde a los valores determinados por el control digital. Esta modificación permite un ajuste preciso y programable de las referencias de voltaje, eliminando la necesidad de ajustes mecánicos mejorando la precisión y flexibilidad del sistema.
\begin{figure}[H]
    \centering
    \includegraphics[scale=0.3]{./imagenes/Eliminada1.jpg}
    \caption{Sección de referencia de tensión.}
    \label{F:estructura_archivos}
\end{figure}

\subsection{Modificación de uso de Potenciómetros Digitales MCP4661}
Originalmente, los potenciómetros digitales MCP4661 se utilizaban para establecer una referencia de voltaje que comandaba los transistores, definiendo tanto la tensión como la corriente sobre la carga. Sin embargo, con la incorporación del DAC, esta función ya no es necesaria. En su lugar, los potenciómetros digitales ahora se utilizan para establecer una referencia de tensión destinada a un circuito de protección analógica contra cortocircuitos. Esta reasignación permite una respuesta inmediata para proteger la carga, evitando los retrasos inherentes a los cálculos y actualizaciones de salida necesarios en un sistema de control digital.
\begin{figure}[H]
    \centering
    \includegraphics[scale=0.2]{./imagenes/potenciometro_digital.jpg}
    \caption{Potenciómetros digital MCP4661.}
    \label{F:potenciometro_digital}
\end{figure}

\subsection{Eliminación del circuito de medición externo}
Dado que la fuente de alimentación ahora cuenta con una pantalla integrada que muestra en tiempo real los valores de tensión y corriente, el circuito dedicado a la conexión de un voltímetro-amperímetro digital se ha considerado innecesario y, por lo tanto, ha sido eliminado. Esta simplificación reduce la complejidad del diseño y el número de componentes necesarios reduciendo los costos constructivos de la fuente.
\begin{figure}[H]
    \centering
    \includegraphics[scale=0.2]{./imagenes/voltimetro_amperimetro.jpg}
    \caption{Conexión voltímetro/amerimetro.}
    \label{F:voltimetro_amperimetro}
\end{figure}

\subsection{Modificación del Circuito de Acople y Desacople de Carga}
Se ha reducido considerablemente el circuito de disparo del optoacoplador, aprovechando las capacidades proporcionadas por el Arduino Nano para establecer un pin en estado alto.
\begin{figure}[H]
    \centering
    \includegraphics[scale=0.3]{./imagenes/conexion_carga.jpg}
    \caption{Acople y desacople de carga.}
    \label{F:conexion_carga}
\end{figure}

\subsection{Simplificación del Circuito Indicador de Modo de Operación}
La pantalla integrada también cumple la función de indicar el modo de operación, eliminando la necesidad de un circuito adicional dedicado a esta tarea. Esto no solo simplifica el diseño del sistema, sino que también mejora la usabilidad al centralizar toda la información relevante en un solo lugar.
\begin{figure}[H]
    \centering
    \includegraphics[scale=0.5]{./imagenes/modo_operacion.jpg}
    \caption{Sección de referencia de tensión.}
    \label{F:modo_operacion}
\end{figure}

\subsection{Integración del Circuito con NodeMCU ESP-32S}
Dado que no se requiere una conexión inalámbrica según las especificaciones del circuito, se ha prescindido del microcontrolador ESP con módulo Wi-Fi para visualizar la información en una computadora.
\begin{figure}[H]
    \centering
    \includegraphics[scale=0.5]{./imagenes/ESP_32S.jpg}
    \caption{Circuito registrador de datos.}
    \label{F:ESP_32S}
\end{figure}

\subsection{Encoders rotativos}
El uso de un teclado numérico hace que este elemento se vuelva totalmente innecesario para estas aplicaciones dado a que el objetivo de la fuente es que sea totalmente digital evitando el ajuste manual de las magnitudes. Sin embargo no habría problema en implementar este elemento en paralelo en caso de un nuevo diseño para seteo de magnitudes, esto debido a una cuestión tradicional ya que la gran mayoría de los usuarios estan acostumbrado a setear manualmente las fuentes de tensión. 
\begin{figure}[H]
    \centering
    \includegraphics[scale=0.5]{./imagenes/encoder_rotativo.jpg}
    \caption{Encoder rotativos.}
    \label{F:encoder_rotativo}
\end{figure}

Estas modificaciones han permitido no solo modernizar la fuente de alimentación, sino también mejorar su funcionalidad y eficiencia mediante la incorporación de tecnología digital y la simplificación de circuitos redundantes. En las secciones siguientes, se detallarán en profundidad cada uno de estos cambios y su impacto en el rendimiento general del equipo






% ----------------------
  \chapter{Uso de esta plantilla en \LaTeX}

En este capítulo, en gran parte, se demostrará con ejemplos el uso de esta plantilla, y en general el uso de \LaTeX. Algunas cosas, como la estructura de archivos de esta plantilla, requieren cierta explicación, pero siempre que pueda evitarse, simplemente se utilizará un ejemplo de código y el resultado del compilado del mismo. \cite{Instruments1978}

Las cuestiones básicas sobre el uso de \LaTeX, más bien que ser explicadas de forma tediosa en este texto, se recomienda al alumno buscar videotutoriales en línea. Como ejemplo considere esta lista de reproducción \cite{tutorial_latex}.

La forma en la que se encuentra desarrollado este capítulo es sencillo y no requiere mayor explicación que la recién brindada.

% ----------------------
\section{Estructura del Informe}

En la figura~\ref{F:estructura_archivos} se puede ver la estructura de archivos. En la carpeta \entreComillas{contenido}, se encuentran los archivos que componen cada capítulo del informe; en \entreComillas{apéndices}, se encuentran los apéndices; en \entreComillas{imagenes}, las imágenes y gráficos utilizados, sea el formato que sea, \texttt{.png}, \texttt{.jpg}, \texttt{.pdf}, o cualquier otro; en \entreComillas{bibliografia} se encuentra el archivo \texttt{bibliografia.bib}, donde están especificadas todas las referencias utilizadas en el informe; y en \entreComillas{codigo}, se encuentra el código fuente de programas que se hayan incluido al informe.

Luego se tienen otros archivos que están fuera de las carpetas. Uno de ellos es el documento \texttt{proyectoelectronico.cls}, el cual es una clase donde se define el estilo de la plantilla; y el otro es \texttt{proyecto.tex}, el cual es el documento principal, el cual se compila para generar el archivo \texttt{.pdf} del informe.

\clearpage
\begin{figure}
    \centering
    \includegraphics[scale=0.8]{./imagenes/estructura_de_archivos.png}
    \caption{Estructura de archivos de la plantilla.}
    \label{F:estructura_archivos}
\end{figure}



\clearpage
\subsection{Datos generales}

Los datos de la portada y la página de aprobación se ingresan en \texttt{proyecto.tex}. A continuación, se muestra exactamente donde se deben ingresar los datos.

\footnotesize
\begin{lstlisting}[language=TeX, numbers=none]
% Titulo del proyecto
\titulo{Colocar aqui el nombre del proyecto}

% Autor (nombre y carne)
\autoruno{Nombre del primer integrante}
\carneuno{Legajo del primer integrante} % legajo
\emailuno{Email del primer integrante}

\autordos{Nombre del segundo integrante}
\carnedos{Legajo del segundo integrante} % legajo
\emaildos{Email del segundo integrante}

%\autortres{Nombre del tercer integrante}
%\carnetres{Legajo del tercer integrante} % legajo
%\emailtres{Email del tercer integrante}

% Profesor(a) guia
\guia{Nombre del profesor tutor}

% Profesores lectores 
\lectorA{Nombre del primer profesor lector}
\lectorB{Nombre del segundo profesor lector}

% Fecha de entrega del trabajo escrito
\mes{11}	% Numero del mes
\ano{2022}	% Formato AAAA
\end{lstlisting}
\normalsize

Como puede apreciarse existen campos para incluir un tercer integrante al grupo, en caso de ser necesario, descomentar los campos y rellenarlos. Para poder visualizar al tercer integrante en la carátula, página de aprobación, resumen y abstract, debemos dirigirnos a \texttt{proyectoelectronico.cls} y descomentar algunas líneas; las mismas se muestran a continuación. Se recomienda utilizar el buscador (Ctrl + F) para encontrarlas rápidamente.

Estas líneas son de la carátula, y deben ser descomentadas si esperamos que el tercer integrante aparezca en la misma.

\footnotesize
\begin{lstlisting}[language=TeX, numbers=none]
%	\vskip 0.8em
%	\large\bfseries \@autortres \\
%	\vskip 0.1em
%	\large\bfseries \@emailtres \\
\end{lstlisting}
\normalsize

\clearpage
Estas líneas son de la hoja de aprobación, y deben ser descomentadas si esperamos que el tercer integrante aparezca en la misma.

\footnotesize
\begin{lstlisting}[language=TeX, numbers=none]
%	\large\bfseries \@autortres \\
%	\vskip 0.1em
%	\large\bfseries \@emailtres \\
%	\vskip 0.1em
%	\large\bfseries \@carnetres \\
\end{lstlisting}
\normalsize

Esta última línea se repite dos veces, una para el resumen y otra vez para el abstract. La misma debe ser descomentada en ambas partes si esperamos que el tercer integrante aparezca correctamente en ellas.

\footnotesize
\begin{lstlisting}[language=TeX, numbers=none]
%		\large\bfseries\@autortres
\end{lstlisting}
\normalsize

\subsection{Resumen}

Para escribir el resumen\footnote{Recordar que el resumen es una de las cosas que se termina por último, hay que tener el informe escrito casi en su totalidad para saber que escribir en el resumen. En esta sección simplemente se describe como modificarlo, pero debe realizarse por último.} es necesario ir al archivo \texttt{./contenido/resumen.tex}. El contenido actual del mismo se muestra a continuación.
\footnotesize
\begin{lstlisting}[language=TeX]
% EL RESUMEN
% ----------

\begin{resumen}{Aqui, van, las, palabras, claves, separadas, por, comas}

\lipsum[1-2]

\end{resumen}
\end{lstlisting}
\normalsize

Aquí debe agregarse las palabras claves separadas por coma, y luego escribir el resumen en sí. Para ello es necesario eliminar la línea con el comando \texttt{$\backslash$lipsum[1-2]}, el mismo es simplemente un comando para generar texto aleatorio con el objetivo de rellenar plantillas y ver como quedarían si tuviesen texto, por lo tanto, debe eliminarse antes de escribir el resumen.

\clearpage
\subsection{Abstract}

El abstract es simplemente una traducción del resumen y para escribirlo es necesario ir al archivo \texttt{./contenido/abstract.tex}. El contenido actual del mismo se muestra a continuación.
\footnotesize
\begin{lstlisting}[language=TeX]
% EL RESUMEN EN INGLES
% --------------------

\begin{theabstract} {Here goes the translated title of the project} {Here, goes, the, keywords, separated, by, commas}

\lipsum[1-2]

\end{theabstract}
\end{lstlisting}
\normalsize

Su modificación es similar a la del resumen, con la diferencia de que aquí el entorno \entreComillas{theabstract} recibe dos parámetros, primero el título traducido y luego las palabras claves, por supuesto, también en inglés.

Al igual que el resumen, para escribir el abstract es necesario eliminar la línea con el comando \texttt{$\backslash$lipsum[1-2]} antes.

\subsection{Agradecimientos}

Para agregar los agradecimientos es necesario simplemente modificar el archivo: \\ \texttt{./contenido/agradecimientos.tex}.

\subsection{Nomenclatura}

Para modificar las nomenclaturas utilizadas debe modificarse el archivo: \\ \texttt{./contenido/nomenclatura.tex}.

Para ello se utiliza el comando \texttt{$\backslash$nomenclature\{\}\{\}}, el mismo recibe dos parámetros, el primero de ellos es la abreviatura o palabra, y el segundo su significado. En el archivo actual se encuentran muchos ejemplos, los cuales deben ser quitados si no son utilizados y agregar los que sí aplican al informe.

\subsection{Índices}

En el archivo \texttt{proyecto.tex} se encuentra especificado cómo se imprime la tabla de contenidos

\footnotesize
\begin{lstlisting}[language=TeX, numbers=none]
% 5. TABLAS DE CONTENIDO, FIGURAS Y TABLAS
\tableofcontents
\listoffigures
\listoffotos
\listoftables
\lstlistoflistings
\end{lstlisting}
\normalsize

Como hay muchos índices separados, deberíamos retirar los que no utilizamos. Por ejemplo, si no incluimos el código de ningún programa, deberíamos no incluir el índice de listados, es decir, quitar el comando \texttt{$\backslash$lstlistoflistings}.

\clearpage
\subsection{Modificaciones extras}

La idea de utilizar una plantilla es precisamente poder simplemente escribir el documento sin tener que molestarse con los detalles de formateo del mismo. Sin embargo, si es de interés para el alumno realizar modificaciones, mencionamos que en el archivo \texttt{proyectoelectronico.cls} se encuentra formateado la portada:

\footnotesize
\begin{lstlisting}[language=TeX, numbers=none]
% 1. Formato de portada
% ---------------------

\newcommand{\portada}{
...
}
\end{lstlisting}
\normalsize

La hoja de aprobación:

\footnotesize
\begin{lstlisting}[language=TeX, numbers=none]
% 2. Formato de la hoja de aprobacion
% -----------------------------------

\newcommand{\aprobacion}{
...
}
\end{lstlisting}
\normalsize

El resumen:

\footnotesize
\begin{lstlisting}[language=TeX, numbers=none]
% 3. Formato del resumen
% ----------------------

\NewDocumentEnvironment{resumen}{ m }
{
...
}
\end{lstlisting}
\normalsize

El abstract:

\footnotesize
\begin{lstlisting}[language=TeX, numbers=none]
% 4. Formato del abstract
% -----------------------

\NewDocumentEnvironment{theabstract}{ m m }
{
...
}
\end{lstlisting}
\normalsize

La hoja de agradecimientos (llamado reconocimientos en el código):

\footnotesize
\begin{lstlisting}[language=TeX, numbers=none]
% 5. Formato de los reconocimientos
% ---------------------------------

\NewDocumentEnvironment{reconocimiento}{ m }
{
...
}
\end{lstlisting}
\normalsize


\clearpage
\subsection{Agregar un capítulo}

Para agregar un capítulo nuevo creamos un archivo dentro de la carpeta \texttt{./contenido/} haciendo click derecho y seleccionando \entreComillas{Archivo nuevo}, como puede apreciarse en la figura~\ref{F:crear_nuevo_capitulo}.

\begin{figure}[ht]
  \centering
  \includegraphics[width=0.45\textwidth]{./imagenes/crear_nuevo_capitulo.png}
  \caption{Creación de nuevo capítulo}
  \label{F:crear_nuevo_capitulo}
\end{figure}

Le asignamos al archivo un nombre apropiado y dentro del mismo utilizamos  \texttt{$\backslash$chapter\{\}}, para definir el capítulo y su nombre.

\footnotesize
\begin{lstlisting}[language=TeX, numbers=none]
\chapter{Nombre del nuevo capitulo}

% Texto de mi nuevo capitulo...
\end{lstlisting}
\normalsize

Dentro de este archivo es donde escribiremos nuestro capítulo. Pero para incluirlo al informe primero debemos dirigirnos al archivo \texttt{proyecto.tex}, y agregarlo con el comando \texttt{$\backslash$input\{\}}, teniendo en cuenta la posición del mismo respecto de los otros capítulos. Como puede observarse en la captura de la figura~\ref{F:nuevo_capitulo}, el mismo se agregó justo seguido de las conclusiones, pero podría ser puesto en otro orden.
\clearpage
\begin{figure}[ht]
  \centering
  \includegraphics[width=0.85\textwidth]{./imagenes/nuevo_capitulo.png}
  \caption{Creación de nuevo capítulo}
  \label{F:nuevo_capitulo}
\end{figure}

\clearpage
\section{Figuras}

\footnotesize
\begin{lstlisting}
En la figura~\ref{F:diag_bloq_sistema_minimo} puede observarse el diagrama de bloques del sistema minimo.

\begin{figure}[ht]
  \centering
  \includegraphics[width=0.9\textwidth]{./imagenes/diag_bloq_sistema_minimo.pdf}
  \caption{Diagrama de bloques del sistema minimo}
  \label{F:diag_bloq_sistema_minimo}
\end{figure}
\end{lstlisting}
\normalsize

En la figura~\ref{F:diag_bloq_sistema_minimo} puede observarse el diagrama de bloques del sistema mínimo.

\begin{figure}[ht]
  \centering
  \includegraphics[width=0.9\textwidth]{./imagenes/diag_bloq_sistema_minimo.pdf}
  \caption{Diagrama de bloques del sistema mínimo}
  \label{F:diag_bloq_sistema_minimo}
\end{figure}

\clearpage
\footnotesize
\begin{lstlisting}
\begin{figure}[ht!]
\centering
\subfloat[Transistor (texto que aparece en el indice)][Transistor en encapsulado TO-220]{
	\includegraphics[width=0.2\textwidth]{./imagenes/transistor.jpg}
	\label{F:subfig1}}
\qquad
\subfloat[LED][LED blanco de baja potencia]{
	\includegraphics[width=0.2\textwidth]{./imagenes/led.jpg}
	\label{F:subfig2}}
\\
\subfloat[Fotoconductor][Fotoconductor]{
	\includegraphics[width=0.2\textwidth]{./imagenes/fotoconductor.jpg}
	\label{F:subfig3}}
\qquad
\subfloat[Circuito integrado][Circuito integrado en encapsulado DIP-8]{
	\includegraphics[width=0.2\textwidth]{./imagenes/integrado.jpg}
	\label{F:subfig4}}
\caption{Una figura con varias subfiguras, utilizando el paquete \texttt{subfig}}
\label{F:subfiguras}
\end{figure}
\end{lstlisting}
\normalsize


\begin{figure}[ht!]
\centering
\subfloat[Transistor (texto que aparece en el índice)][Transistor en encapsulado TO-220]{
	\includegraphics[width=0.2\textwidth]{./imagenes/transistor.jpg}
	\label{F:subfig1}}
\qquad
\subfloat[LED][LED blanco de baja potencia]{
	\includegraphics[width=0.2\textwidth]{./imagenes/led.jpg}
	\label{F:subfig2}}
\\
\subfloat[Fotoconductor][Fotoconductor]{
	\includegraphics[width=0.2\textwidth]{./imagenes/fotoconductor.jpg}
	\label{F:subfig3}}
\qquad
\subfloat[Circuito integrado][Circuito integrado en encapsulado DIP-8]{
	\includegraphics[width=0.2\textwidth]{./imagenes/integrado.jpg}
	\label{F:subfig4}}
\caption{Una figura con varias subfiguras, utilizando el paquete \texttt{subfig}}
\label{F:subfiguras}
\end{figure}

\clearpage
\section{Fotografías}

A pedido de la cátedra se incluyó un entorno distinto al de figuras para presentar fotografías. El mismo se utiliza de igual manera que el entorno de figuras, con la única diferencia de que este entorno se llama \entreComillas{foto}, como se muestra a continuación.

\footnotesize
\begin{lstlisting}
En la fotografia~\ref{F:foto_sistema_med} puede verse el sistema medio construido.

\begin{foto}[ht]
  \centering
  \includegraphics[width=0.9\textwidth]{./imagenes/foto_sistema_med.jpg}
  \caption{Sistema medio construido}
  \label{F:foto_sistema_med}
\end{foto}
\end{lstlisting}
\normalsize

En la fotografía~\ref{F:foto_sistema_med} puede verse el sistema medio construido.

\begin{foto}[ht]
  \centering
  \includegraphics[width=0.9\textwidth]{./imagenes/foto_sistema_med.jpg}
  \caption{Sistema medio construido}
  \label{F:foto_sistema_med}
\end{foto}



\clearpage
\section{Tablas}

A continuación, se muestran algunas tablas como ejemplo. Existen páginas para crear tablas de \LaTeX~de forma rápida y sencilla, como por ejemplo \cite{tablas_latex_online}; el cual se utilizó en varias ocasiones.


\begin{table}[ht!]
\caption{Numeración decimal, binaria y hexadecimal}
\label{T:dec_bin_hex}
\begin{tabular}{|c|c|c|}
\hline
Decimal & Binario & Hexadecimal \\ \hline \hline
0       & 0000    & 0           \\ \hline
1       & 0001    & 1           \\ \hline
2       & 0010    & 2           \\ \hline
3       & 0011    & 3           \\ \hline
4       & 0100    & 4           \\ \hline
5       & 0101    & 5           \\ \hline
6       & 0110    & 6           \\ \hline
7       & 0111    & 7           \\ \hline
8       & 1000    & 8           \\ \hline
9       & 1001    & 9           \\ \hline
10      & 1010    & A           \\ \hline
11      & 1011    & B           \\ \hline
12      & 1100    & C           \\ \hline
13      & 1101    & D           \\ \hline
14      & 1110    & E           \\ \hline
15      & 1111    & F           \\ \hline
\end{tabular}
\end{table}


% Please add the following required packages to your document preamble:
% \usepackage{multirow}
\begin{table}[ht!]
\caption{Códigos de estado}
\label{T:state_codes}
\begin{tabular}{|l|ll|}
\hline
\multicolumn{1}{|c|}{\multirow{2}{*}{State Type}} & \multicolumn{2}{l|}{State Code} \\ \cline{2-3} 
\multicolumn{1}{|c|}{}                            & \multicolumn{1}{l|}{SC1}  & SC0 \\ \hline
S0 (Fetch)                                        & \multicolumn{1}{l|}{L}    & L   \\ \hline
S1 (Execute)                                      & \multicolumn{1}{l|}{L}    & H   \\ \hline
S2 (DMA)                                          & \multicolumn{1}{l|}{H}    & L   \\ \hline
S3 (Interrupt)                                    & \multicolumn{1}{l|}{H}    & H   \\ \hline
\end{tabular}
\end{table}


\clearpage

La tabla~\ref{T:bom_sis_final} es un ejemplo de cómo realizar una tabla que es tan grande que se extiende en varias páginas. La misma tiene un encabezado que se repite en cada página, de manera que no es necesario ir al comienzo de la tabla para ver a qué corresponde cada columna, dado que el encabezado está presente al inicio de cada página.

\footnotesize
\begin{longtable}{|P{0.08\textwidth}|P{0.11\textwidth}|P{0.30\textwidth}|P{0.14\textwidth}|P{0.15\textwidth}|}
\caption{Lista de componentes del sistema final} % needs to go inside longtable environment
\label{T:bom_sis_final}
\\
\hline
Cantidad & Etiqueta \newline
           Identificador            & Descripción                               & Fabricante            & Número de parte       \\ \hline \hline \endfirsthead
\hline
Cantidad & Etiqueta \newline
           Identificador            & Descripción                               & Fabricante            & Número de parte       \\ \hline \hline \endhead
1       & U1                        & Microprocesador                           & RCA                   & CDP1802ACE            \\ \hline
1       & U2                        & Latch de 8 bits                           & Philips               & 74HC573N              \\ \hline
1       & U3                        & Memoria EEPROM 32k x 8                    & XICOR                 & X28C256D-25           \\ \hline
1       & U4                        & Memoria EEPROM Serial I2C 128k x 8        & Microchip             & 24LC1025              \\ \hline
1       & U5                        & Time keeping RAM 32K x8                   & Dallas                & DS1744-070            \\ \hline
1       & U6                        & Memoria RAM 32k x 8                       & HYUNDAI               & HY62256ALP-10         \\ \hline
1       & U7                        & Memoria FRAM Serial I2C 8k x 8            & Fujitsu Semiconductors & MB85RC64A            \\ \hline
1       & U8                        & Conversor analógico-digital con interfaz I2C & Maxim Integrated   & MAX127                \\ \hline
1       & U9                        & Regulador de tensión 12~\si{\volt}        & Motorola              & 7812CT                \\ \hline
1       & U10                       & Regulador de tensión 5~\si{\volt}         & ST Microelectronics   & L7805CV               \\ \hline
1       & U11                       & High Precision Operational Amplifier      & Burr Brown            & OPA4277PA             \\ \hline
1       & U12                       & Regulador de tensión -12~\si{\volt}       & ST Microelectronics   & L7912CV               \\ \hline
5       & U13, U14, U15, U16, U28   & Cuadruple compuerta NAND de dos entradas  & Texas Instrument      & SN74HC00N             \\ \hline
1       & U17                       & Cuadruple compuerta NAND de dos entradas con Schmitt-Trigger & Texas Instrument & SN74HC132N \\ \hline
2       & U18,U30                   & Cuadruple compuerta NOR de dos entradas   & Texas Instrument      & SN74HC02N             \\ \hline
1       & U19                       & Decodificador Multiplexador 3 a 8 líneas  & Texas Instrument      & SN74HC138N            \\ \hline
1       & U20                       & CMOS Programmable Peripheral Interface    & Intersil              & CP82C55A-5Z           \\ \hline
1       & U21                       & Regulador programable shunt               & Fairchild             & LM336Z25              \\ \hline
1       & U24                       & Doble flip-flop tipo D con Set y Reset    & Texas Instrument      & SN74HC74N             \\ \hline
1       & U25                       & Contador binario de 4 bits                & Fairchild Semiconductor & MM74HC161N          \\ \hline
1       & U29                       & Doble flip-flop JK con Set y Reset        & National Semiconductors & MM74HC73N           \\ \hline
1       & U31                       & Registro de desplazamiento de 8 bits      & Texas Instrument      & SN74HC595N            \\ \hline
1       & U32                       & Registro de desplazamiento de 8 estados   & Motorola              & MC14014BCP            \\ \hline
1       & X2                        & XO-22BE-4MHz                              &   \listaVacio                    & \listaVacio                      \\ \hline
1       & XTAL1                     & Cristal \num{4}~\si{\mega\hertz}          & CQ Electronics        & \listaVacio                      \\ \hline
1       & SW1                       & Llave DPDT 6 pines                        & \listaVacio                      & \listaVacio                      \\ \hline
3       & SW2, SW3, SW4             & Pulsador 6mm~x~4,3mm                      &  \listaVacio                     &  \listaVacio                     \\ \hline
5       & T2, T3, T4, T5, T6        & Transistor BJT NPN                        & Motorola              & P2N2222A               \\ \hline
1       & RV1                       & Preset 10~\si{\kilo\ohm}                  &  \listaVacio                     &  \listaVacio                     \\ \hline
1       & D1                        & Puente rectificador de onda completa B380R &            \listaVacio           &  \listaVacio                     \\ \hline
6       & D2, D3, D4, D5, D6, D7    & LED 5~mm                                  &\listaVacio                       &   \listaVacio                    \\ \hline
2       & D8, D9                    & 1N914                                     &\listaVacio                       &  \listaVacio                     \\ \hline
1       & F1                        & Fusible 10~mm 1~\si{\ampere}              &\listaVacio                       &  \listaVacio                     \\ \hline
8       & R1, R2, R3, R5, R21, R22, R23, R27
                                    & Resistor 47~\si{\kilo\ohm} 1/8~\si{\watt} & \listaVacio                      &   \listaVacio                    \\ \hline
1       & R4                        & Resistor 10~\si{\mega\ohm} 1/8~\si{\watt} &  \listaVacio                     &   \listaVacio                    \\ \hline
1       & R6                        & Resistor 330~\si{\ohm} 1/4~\si{\watt}     &  \listaVacio                     &   \listaVacio                    \\ \hline
2       & R7, R8                    & Resistor 1~\si{\kilo\ohm} 1/8~\si{\watt}  &  \listaVacio                     &    \listaVacio                   \\ \hline
3       & R9, R10, R11              & Resistor 220~\si{\ohm} 1/4~\si{\watt}     &    \listaVacio                   &    \listaVacio                   \\ \hline
16      & R12, R13, R14, R15, R16, R17, R18, R38, R39, R40, R41, R46, R47, R56, R57, R60
                                    & Resistor 10~\si{\kilo\ohm} 1/8~\si{\watt} & \listaVacio                      &   \listaVacio                    \\ \hline
5       & R19, R20, R24, R25, R26   & Resistor 100~\si{\kilo\ohm} 1/8~\si{\watt}&  \listaVacio                     &   \listaVacio                    \\ \hline
4       & R28, R30, R36, R52        & Resistor 100~\si{\ohm} 1/4~\si{\watt}     &  \listaVacio                     &   \listaVacio                    \\ \hline
4       & R29, R31, R37, R53        & Resistor 150~\si{\ohm} 1/4~\si{\watt}     &  \listaVacio                     &   \listaVacio                    \\ \hline
8       & R32, R33, R34, R35, R42, R43, R54, R55
                                    & Resistor 390~\si{\kilo\ohm} 1/8~\si{\watt}&  \listaVacio                     & \listaVacio                      \\ \hline
8       & R44, R45, R48, R49, R50, R51, R58, R59
                                    & Resistor 1~\si{\mega\ohm} 1/8~\si{\watt}  &  \listaVacio                     & \listaVacio                      \\ \hline
2       & C2, C3                    & Capacitor cerámico \num{30}~\si{\pico\farad} 50~\si{\volt}
                                                                                &   \listaVacio                    & \listaVacio                      \\ \hline
14       & C5, C7, C8, C29, C30, C31, C14, C15, C18, C32, C33, C39, C40, C43
                                    & Capacitor cerámico \num{0.10}~\si{\micro\farad} 50~\si{\volt}
                                                                                &    \listaVacio                   &   \listaVacio                    \\ \hline
1       & C11, C17                  & Capacitor electrolítico 4700~\si{\micro\farad} 25~\si{\volt}
                                                                                &     \listaVacio                  &   \listaVacio                    \\ \hline
2       & C12, C16                  & Capacitor cerámico \num{0.33}~\si{\micro\farad} 50~\si{\volt}
                                                                                &   \listaVacio                    &  \listaVacio                     \\ \hline
1       & C13                       & Capacitor electrolítico 10~\si{\micro\farad} 25~\si{\volt}
                                                                                &   \listaVacio                    &  \listaVacio                     \\ \hline
1       & C19                       & Capacitor electrolítico \num{0.01}~\si{\micro\farad} 25~\si{\volt}
                                                                                &     \listaVacio                  &  \listaVacio                     \\ \hline
1       & C20                       & Capacitor electrolítico \num{4.7}~\si{\micro\farad} 25~\si{\volt}
                                                                                &   \listaVacio                    &  \listaVacio                     \\ \hline
6       & J1, J2, J33, J48, J49, J50& 1x3 Pines Conectores Macho \num{2.54}~mm  &   \listaVacio                    &  \listaVacio                     \\ \hline
38      & J3, J4, J5, J6, J7, J8, J9, J10, J11, J12, J13, J14, J15, J16, J17, J18, J19, J20, J21, J22, J23, J24, J25, J26, J27, J28, J29, J30, J31, J36, J40, J41, J42, J45, J46, J51, J52, J53
                                    & Pin conector macho (1 \textit{Male Header Pin}) & \listaVacio                      &   \listaVacio                    \\ \hline
3       & J32, J35, J38             & 1x6 Pines Conectores Macho \num{2.54}~mm  & \listaVacio                      & \listaVacio                      \\ \hline
3       & J34, J37, J39             & 1x5 Pines Conectores Macho \num{2.54}~mm  & \listaVacio                      &   \listaVacio                    \\ \hline
2       & J43, J44                  & Bornera 3 pines \num{5.08}~mm             &  \listaVacio                     &  \listaVacio                     \\ \hline
2       & J47, J54                  & 1x12 Pines Conectores Macho \num{2.54}~mm &   \listaVacio                    &  \listaVacio                     \\ \hline
1       & J55                       & 1x2 Pines Conectores Macho \num{2.54}~mm  &   \listaVacio                    &   \listaVacio                    \\ \hline
1       & No aplica                 & Módulo conversor USB-Serial               & Future Technology Devices International & FT232RL \\ \hline
1       & No aplica                 & Módulo adaptador tarjeta micro SD         & \listaVacio  & \listaVacio \\ \hline
\end{longtable}
\normalsize

\clearpage
\section{Números y Unidades}

A continuación, se muestra como escribir unidades de forma correcta.


\begin{lstlisting}[language=TeX, numbers=none]
Las unidades se escriben utilizando el paquete siunitx. Puede ser asi: \SI{2.2}{\kilo\ohm}, o tambien ser asi: \num{2.2} \si{\kilo\ohm}.
\end{lstlisting}
\normalsize

Si compilamos esto, obtenemos:

Las unidades se escriben utilizando el paquete siunitx. Puede ser así: \SI{2.2}{\kilo\ohm}, o también ser así: \num{2.2} \si{\kilo\ohm}.

\vspace{1cm}

Pero es importante utilizar correctamente el espacio de no separación \textasciitilde~(que es el carácter 126 del código ASCII) para separar el número de la unidad, si se escriben por separado. De esta manera se evita que en los saltos de línea se separe el número de la unidad. Reescribamos lo anterior pero esta vez con un espacio de no separación.


\begin{lstlisting}[language=TeX, numbers=none]
Las unidades se escriben utilizando el paquete siunitx. Puede ser asi: \SI{2.2}{\kilo\ohm}, o tambien ser asi: \num{2.2}~\si{\kilo\ohm}.
\end{lstlisting}
\normalsize

Si compilamos esto, obtenemos:

Las unidades se escriben utilizando el paquete siunitx. Puede ser asi: \SI{2.2}{\kilo\ohm}, o tambien ser asi: \num{2.2}~\si{\kilo\ohm}.

\vspace{1cm}

Como vemos, ahora no se separó el número de la unidad.

\clearpage
\section{Ecuaciones}

\footnotesize
\begin{lstlisting}
En la ecuacion~\eqref{eq:ecuacion_1} se encuentra la formula de Euler.

\begin{equation}
\label{eq:ecuacion_1}
 e^{jx} = \cos{x} + j \sin{x}
\end{equation}
\end{lstlisting}
\normalsize

En la ecuación~\eqref{eq:ecuacion_1} se encuentra la fórmula de Euler.

\begin{equation}
\label{eq:ecuacion_1}
 e^{jx} = \cos{x} + j \sin{x}
\end{equation}

\footnotesize
\begin{lstlisting}
\begin{equation}
u(x) = 
  \begin{cases} 
   \exp{x} & \text{si } x \geq 0 \\
   1       & \text{si } x < 0
  \end{cases}
\end{equation}
\end{lstlisting}
\normalsize

\begin{equation}
u(x) = 
  \begin{cases} 
   \exp{x} & \text{si } x \geq 0 \\
   1       & \text{si } x < 0
  \end{cases}
\end{equation}

\footnotesize
\begin{lstlisting}
\begin{subequations}
    \begin{equation}
    r_1^2 = (h_T-h_R)^2+d^2
    \label{eq: r1_tierra_plana}
    \end{equation}
    \begin{equation}
    r_2^2 = (h_T+h_R)^2+d^2
    \label{eq: r2_tierra_plana}
    \end{equation}
\end{subequations}
\end{lstlisting}
\normalsize

\begin{subequations}
    \begin{equation}
    r_1^2 = (h_T-h_R)^2+d^2
    \label{eq: r1_tierra_plana}
    \end{equation}
    \begin{equation}
    r_2^2 = (h_T+h_R)^2+d^2
    \label{eq: r2_tierra_plana}
    \end{equation}
\end{subequations}


\clearpage
\section{Código fuente}

Código fuente puede ser ingresado de la siguiente manera.

\footnotesize
\begin{lstlisting}
En el listado~\ref{L:codigo_ejemplo} puede verse en codigo de ejemplo en Octave.

\footnotesize
\lstinputlisting[language=Octave, caption = {Codigo de ejemplo en Octave}, label = {L:codigo_ejemplo}]{codigo/codigo_ejemplo.m}
\normalsize
\end{lstlisting}
\normalsize

En el listado~\ref{L:codigo_ejemplo} puede verse en código de ejemplo en Octave.

\footnotesize
\lstinputlisting[language=Octave, caption = {Código de ejemplo en Octave}, label = {L:codigo_ejemplo}]{codigo/codigo_ejemplo.m}
\normalsize

También es posible definir un resaltado de sintaxis personalizado. Fue necesario definir uno para el lenguaje ensamblador del microprocesador CDP1802; así que presentamos el mismo como ejemplo. El archivo que define la sintaxis se encuentra en \texttt{./codigo/definiciondeASM.tex}, y para incluirlo debemos dirigirnos a \texttt{proyectoelectronico.cls} y añadir el mismo con el comando \texttt{$ \backslash$input\{\}}, como se muestra a continuación (debe ser luego de haber incluido el paquete \entreComillas{listings}).

\footnotesize
\begin{lstlisting}
\usepackage{listings}
%% este código define el estilo para el lenguaje assembler del CDP1802

\lstdefinelanguage{CDP1802}{
    morekeywords=[1]{LDN, LDA, LDX, LDXA, LDI, STR, STXD,
        INC, DEC, IRX, GLO, PLO, GHI, PHI,
        OR, ORI, XOR, XRI, AND, ANI, SHR, SHCR, RSHR, SHL, SHLC, RSHL,
        ADD, ADI, ADC, ADCI, SD, SDI, SDB, SDBI, SM, SMI, SMB, SMBI,
        BR, NBR, BZ, BNZ, BDF, BPZ, BGE, BNF, BM, BL, BQ, BNQ, B1, BN1, B2, BN2, B3, BN3, B4, BN4,
        LBR, NLBR, LBZ, LBNZ, LBDF, LBNF, LBQ, LBNQ,
        SKP, LSKP, LSZ, LSNZ, LSDF, LSNF, LSQ, LSNQ, LSIE,
        IDL, NOP, SEP, SEX, SEQ, REQ, SAV, MARK, RET, DIS,
        OUT1, OUT2, OUT3, OUT4, OUT5, OUT6, OUT7, INP1, INP2, INP3, INP4, INP5, INP6, INP7, OUT, INP},%
    morekeywords=[2]{r0, r1, r2, r3, r4, r5, r6, r7, r8, r9, r10, r11, r12, r13, r14, r15, ra, rb, rc, rd, re, rf},%
    morekeywords=[3]{org,def,equ,db,dw,include,dseg,cseg,eseg},%
    sensitive=false, % keywords are not case-sensitive
    morecomment=[l]{;}, % l is for line comment
} %

\definecolor{MyDarkGreen}{rgb}{0.0,0.4,0.0} % This is the color used for comments

\lstset{language=CDP1802,
        frame=single, % Single frame around code
        basicstyle=\small\ttfamily, % Use small true type font
        keywordstyle=\color{Blue}\textbf, % Instructions in blue, bold
        keywordstyle=[2]\color{Orange}, % Registers in orange
        keywordstyle=[3]\color{Purple}, % Directives in purple
        commentstyle=\usefont{T1}{pcr}{m}{sl}\color{MyDarkGreen}\small,
        tabsize=4, % 4 spaces per tab
        numbers=left, % Line numbers on left
        firstnumber=1, % Line numbers start with line 1
        numberstyle=\tiny\color{Blue}, % Line numbers are blue and small
        stepnumber=5 % Line numbers go in steps of 5
        } % en este archivo esta la definicion para el estilo de texto en asembler del CDP1802
\end{lstlisting}
\normalsize

Ahora ya podemos utilizar nuestra sintaxis personalizada como se muestra a continuación.

\clearpage

\footnotesize
\begin{lstlisting}
\footnotesize
\lstinputlisting[language=CDP1802, caption = {Rutina de retardo de 1 bit-time}, label = {L:retardo}]{codigo/delay_routine.asm}
\normalsize
\end{lstlisting}
\normalsize

\footnotesize
\lstinputlisting[language=CDP1802, caption = {Rutina de retardo de 1 bit-time}, label = {L:retardo}]{codigo/delay_routine.asm}
\normalsize

\clearpage
\section{Bibliografía}

Los elementos de la bibliografía se encuentran en el archivo \texttt{./bibliografia/bibliografia.bib}, puede abrir el mismo para ver las referencias utilizadas en esta plantilla.


\footnotesize
\begin{lstlisting}
Para citar una referencia se utiliza el comando \cite{plantilla_universidad_de_costa_rica}, y se ingresa la etiqueta de la referencia que deseamos incluir.
\end{lstlisting}
\normalsize

Para citar una referencia se utiliza el comando \cite{plantilla_universidad_de_costa_rica}, y se ingresa la etiqueta de la referencia que deseamos incluir.

Para administrar la bibliografía se recomienda utilizar un programa específico llamado JabRef\cite{jabref}.


\chapter{Conclusiones}

\lipsum[5-6]
\chapter{Estrategia de control}
% ----------------------

\label{C:Formas de control}

\section{Principio de estrategia de control.}


\subsection{Lazo de tensión}

\subsection{Lazo de corriente.}
\chapter{Control digital}
% ----------------------

\label{C:Digitalización y control}

\section{Diagrama de bloques de la etapa digital}
Para la etapa digital se propone el diagrama de bloques de la figura~\ref{F:diagrama_digital}. Se pretende controlar la tensión y corriente de salida mediante el ajuste de las referencias con un teclado numérico, de tal manera que mediante comunicación serie I2C podamos enviar los datos que proporcionan la referencia de tensión y corriente para el lazo de control. A su vez, por el bus I2C se lleva a cabo la lectura de la tensión y corriente de salida mediante un convertidor AD de alta resolución (12 o 16 bits) y los datos procesados se despliegan en un display OLED o LCD. En el display se proporciona la tensión y corriente de salida medidas, la tensión y corriente configurada deseada, y el modo de operación del sistema (CV o CI) así como también si la carga se encuentra conectada o desconectada entre otras funciones. 

\begin{figure} [H]
    \centering
    \includegraphics[scale=0.5]{./imagenes/diagrama_digital.jpg}
    \caption{Estructura de archivos de la plantilla.}
    \label{F:diagrama_digital}
\end{figure}

\section{Componentes de la etapa digital.}

\subsection{dsPIC30F4011. High-Performance, 16-Bit Digital Signal Controllers.}
El dsPIC30F4011 es un controlador digital de señales de 16 bits, reconocido por su alto rendimiento y sus amplias capacidades. Diseñado por Microchip Technology, este dispositivo se destaca por su versatilidad, lo que lo convierte en una opción ideal para una variedad de aplicaciones en el ámbito industrial, comercial y de consumo.
Con su arquitectura avanzada y su conjunto completo de características, el dsPIC30F4011 ofrece un rendimiento excepcional en aplicaciones que requieren un control preciso y eficiente de señales digitales. Su capacidad para manejar operaciones complejas en tiempo real lo hace adecuado para una amplia gama de proyectos, desde sistemas de control hasta aplicaciones de procesamiento de señales. Esta también se destaca por su tamaño compacto y su diseño robusto, lo que lo hace fácil de integrar en una variedad de dispositivos y sistemas electrónicos. Además, su amplio rango de temperatura de funcionamiento y su bajo consumo de energía lo hacen adecuado para aplicaciones en entornos exigentes.

\underline{programación}
La programación del dsPIC30F4011 se lleva a cabo utilizando el software MPLAB IDE v8.91, desarrollado por Microchip Technology Incorporated en los Estados Unidos. MPLAB IDE proporciona un entorno de desarrollo integrado (IDE) que facilita la creación, depuración y programación de aplicaciones para microcontroladores de la familia dsPIC.
Este software ofrece una variedad de herramientas y características que simplifican el proceso de desarrollo de software para el dsPIC30F4011, incluyendo un editor de código, compilador, depurador y simulador. Además, MPLAB IDE es compatible con una amplia gama de dispositivos de Microchip, lo que lo convierte en una opción versátil para los desarrolladores de sistemas embebidos.
El código utilizado en el dsPIC30F4011 está disponible en [insertar referencia aquí], que es un enlace a un repositorio en GitHub. Este repositorio contiene el algoritmo desarrollado en los capítulos anteriores, permitiendo a los interesados examinar y comprender el funcionamiento del sistema implementado. La disponibilidad del código fuente en un repositorio público facilita la colaboración, revisión y mejora continua del proyecto, además de proporcionar una referencia para futuros desarrollos y aplicaciones relacionadas con el dsPIC30F4011.

Protocolo de comunicación.
El protocolo de comunicación es fundamental en el diseño y desarrollo de sistemas embebidos, ya que define la manera en que los dispositivos intercambian información entre sí. En el caso del dsPIC30F4011, se cuenta con diversas opciones de protocolos de comunicación, cada uno con sus propias características y aplicaciones específicas.
Entre los protocolos de comunicación compatibles con el dsPIC30F4011 se encuentran:
SPI™ (Serial Peripheral Interface): Permite la comunicación síncrona entre dispositivos mediante una línea de reloj común y líneas separadas para datos de entrada y salida.
I2C™ (Inter-Integrated Circuit): Proporciona una interfaz de comunicación de bus de dos cables que permite la comunicación entre múltiples dispositivos conectados al mismo bus.
Universal Asynchronous Receiver Transmitter (UART): Permite la comunicación serial asíncrona entre el dsPIC30F4011 y otros dispositivos periféricos.
CAN (Controller Area Network): Es un protocolo de comunicación serial diseñado para aplicaciones de control en tiempo real, especialmente en entornos automotrices e industriales.
Para este proyecto en particular, se optará por emplear el protocolo I2C debido a su compatibilidad con los componentes utilizados en la fuente. La elección de este protocolo se fundamenta en su eficiencia y versatilidad, lo que lo hace idóneo para satisfacer los requisitos de comunicación de este sistema embebido.

\begin{figure}[H]
    \centering 
    \includegraphics[scale=0.5]{./imagenes/mplab.jpg}
    \caption{Logo MPLAB IDE.}
    \label{F:LogoMPLAB}
\end{figure}

\subsection{Teclado de membrana 4x4.}
El teclado de membrana matricial 4x4 autoadhesivo es un dispositivo de entrada que se utiliza comúnmente en aplicaciones electrónicas donde se requiere una interfaz de usuario simple y compacta. Consiste en una delgada lámina de material flexible que contiene una matriz de botones dispuestos en filas y columnas, con un total de 16 botones en este caso particular (4 filas x 4 columnas).
Cada botón en el teclado de membrana está interconectado mediante una disposición de líneas conductoras en la membrana. Estas líneas están organizadas de manera que forman una matriz, permitiendo la detección de la ubicación específica de la tecla presionada. El funcionamiento del teclado de membrana matricial implica un proceso de escaneo continuo de todas las filas y columnas para detectar la presencia de un botón presionado. Cuando un botón se presiona, se cierra un circuito entre la fila y la columna correspondientes, lo que indica al microcontrolador la ubicación de la tecla activada.

\begin{figure}
    \centering [H]
    \includegraphics[scale=0.5]{./imagenes/Teclado Matricial 4x4_2.jpg}
    \caption{Estructura de archivos de la plantilla.}
    \label{F:teclado4x4}
\end{figure}

\subsection{Display OLED SSD1306.}
El display OLED SSD1306 elegido para el proyecto utiliza comunicación I2C y ofrece una resolución de 128x64 píxeles. En la Figura 4.13 se presenta una imagen del display, que opera dentro de un rango de voltaje de 3.3 a 5.5 V, lo cual lo hace compatible con el microcontrolador seleccionado. En esta pantalla se mostrará tanto el menú de funcionamiento los modos de operación como un indicador a tiempo real de las magnitudes registradas. Será el vínculo principal entre el usuario y la fuente.

\begin{figure} [H]
    \centering 
    \includegraphics[scale=0.1]{./imagenes/display.jpg}
    \caption{Display OLED SSD1306.}
    \label{F:display}
\end{figure}

\subsection{Aislador I2C capacitivo.}
El dispositivo a utilizar es un ISO1540 [insertar referencia] el cual cuenta con buffers de entrada y salida que están separados por tecnología de aislamiento capacitivo de Texas Instruments que utiliza una barrera de dióxido de silicio (SiO2). Cuando se utilizan con fuentes de alimentación aisladas, estos dispositivos bloquean voltajes altos, aíslan tierras y evitan corrientes de ruido que puedan ingresar a la tierra local e interferir o dañar circuitos sensibles. Esta tecnología de aislamiento ofrece ventajas en función, rendimiento, tamaño y consumo de energía en comparación con los optoacopladores.
De este modo tendremos la aislación galvánica para separar apropiadamente la parte de potencia de la de control.
\begin{figure}[H]
    \centering
    \includegraphics[scale=0.1]{./imagenes/optoi2c.jpg}
    \caption{Aislador capacitivo I2C ISO1540.}
    \label{F:optoi2c}
\end{figure}

\subsection{Convertidor analógico digital. AD.}
El ADS1115 es un componente crucial en la transición de una fuente de alimentación de corriente continua de analógica a digital. Este dispositivo ofrece una impresionante precisión de 16 bits, junto con una velocidad de muestreo de hasta 860 muestras por segundo a través del protocolo de comunicación I2C. Configurable para operar con cuatro canales de entrada de un solo extremo o dos canales diferenciales, el ADS1115 se destaca por su versatilidad en la medición de señales analógicas en entornos digitales. 
Equipado con un conversor delta-sigma de 16 bits, un comparador programable con salida directa al pin de alerta, y una ganancia ajustable que permite la lectura de hasta 256mV en escala completa, este dispositivo garantiza una captura precisa de los datos analógicos. Su interfaz de comunicación I2C facilita la lectura de datos digitales, mientras que su dirección predeterminada de 0x48 y la disponibilidad de bibliotecas para plataformas como Arduino lo convierten en una opción conveniente y de fácil integración en proyectos electrónicos.
\begin{figure}[H]
    \centering
    \includegraphics[scale=0.1]{./imagenes/ads1115.jpg}
    \caption{Convertidor AD ADS1115.}
    \label{F:ADC}
\end{figure}

\subsection{Modos de funcionamiento.}
El sistema de control de la fuente de alimentación implementa varios modos de funcionamiento para adaptarse a diversas necesidades de aplicación. A continuación, se describen los principales modos de operación:
Modo Tensión:
En este modo, la fuente de alimentación establece inicialmente el valor máximo de tensión deseado. Posteriormente, limita la corriente máxima de umbral que la carga podrá obtener. Este modo es especialmente útil cuando se requiere controlar la tensión suministrada a la carga de manera precisa y garantizar la seguridad del sistema al limitar la corriente máxima.
Modo Corriente:
En el modo de corriente, la fuente de alimentación establece y controla la corriente suministrada a la carga. Este modo es útil en situaciones donde es crítico mantener la corriente dentro de ciertos límites para proteger los componentes de la carga y garantizar su correcto funcionamiento.
Modo Rampa:
El modo de rampa tiene como objetivo generar un aumento gradual y lineal de la tensión suministrada a la carga durante un período de tiempo determinado. Los parámetros configurables en este modo incluyen la tensión final deseada y el tiempo en el cual se alcanzará esta tensión desde un valor inicial de 0V. Este modo es útil en aplicaciones donde se requiere un inicio suave del sistema para evitar sobrecargas o picos de corriente al arrancar la carga.


\chapter{Software programación y ensayo de Control digital}
% ----------------------

\label{C:Software programación y ensayo de Control digital}

\section{Características físicas del microcontrolador}
En esta sección se explorará de manera breve pero sustancial la lógica interna que gobierna el funcionamiento del Arduino Nano y las diversas tareas que este microcontrolador es capaz de realizar. El Arduino Nano, conocido por su versatilidad y eficiencia en proyectos de electrónica y automatización, requiere un software bien planificado para ejecutar sus funciones de manera óptima.

\subsection{Lógica Interna del Arduino Nano}
El Arduino Nano, como todos los microcontroladores de la familia Arduino, opera mediante la ejecución de un conjunto de instrucciones programadas en su memoria flash. Estas instrucciones, escritas en el lenguaje de programación C/C++ utilizando el entorno de desarrollo integrado (IDE) de Arduino, dictan cómo el microcontrolador debe responder a diferentes señales y entradas. En esta sección, se abordarán los conceptos básicos de esta lógica interna, incluyendo el ciclo de procesamiento del Arduino, la gestión de interrupciones, y la manipulación de puertos y registros.

\subsection{Procesamiento de tareas}
El diseño cuenta con 4 funciones básicas.
\begin{enumerate}
    \item Sensado de valores en el ADC.
    De manera constante se estará encuestando al ADC vinculado por la línea I2C qué valor está censando en sus puertos de modo de tener un seguimiento apropiado de los valores presentados registrados en la fuente.    
    \item Procesamiento de tecla por interrupción.
    Al momento de presionar una tecla, el microprocesador se tomará unos segundos para procesar qué acción se ha solicitado en el menú que se verá reflejada de alguna manera en el display.
    \item Actualización de display.
    Vinculado directamente al teclado y al sensado de corriente este mantendrá la actualización del display que nos permitirá ver y obtener un seguimiento acorde en lo que está ocurriendo en la fuente.
    \item Cálculo y actualización de la acción de control.
    Implica la serie de cálculos que determinarán los parámetros de salida que se colocarán en el terminal del dac para que este pueda ser convertido en una referencia de tensión. Esta taréa se llevará a cabo luego de cada finalización de conversión del adc.
\end{enumerate}

\subsection{Pinout de arduino nano}

\begin{figure}[H]
    \centering
    \includegraphics[scale=0.3]{./imagenes/arduino_nano.jpg}
    \caption{Arduino Nano.}
    \label{F:arduino_nano}
\end{figure}

Entradas
\begin{itemize}
    \item Teclado: 8 pines. .
    \item Sensado de tensión y corriente: La corriente de salida se mantiene constante a pesar de los cambios en la carga, la línea o la temperatura.
\end{itemize}

Salidas
\begin{itemize}
    \item Display. Aislador. I2C. DAC.  (SCL; SDA).
    \item Acople Desacople de carga. 1 PIN.
\end{itemize}

\subsection{Protocolo de comunicación.}
El protocolo de comunicación es fundamental en el diseño y desarrollo de sistemas embebidos, ya que define la manera en que los dispositivos intercambian información entre sí. En el caso del arduino nano, se cuenta con diversas opciones de protocolos de comunicación, cada uno con sus propias características y aplicaciones específicas.
Entre los protocolos de comunicación compatibles con el Arduino nano se encuentran:
\begin{itemize}
    \item SPI™ (Serial Peripheral Interface): Permite la comunicación síncrona entre dispositivos mediante una línea de reloj común y líneas separadas para datos de entrada y salida.
    \item I2C™ (Inter-Integrated Circuit): Proporciona una interfaz de comunicación de bus de dos cables que permite la comunicación entre múltiples dispositivos conectados al mismo bus.
    \item Universal Asynchronous Receiver Transmitter (UART): Permite la comunicación serial asíncrona entre el dsPIC30F4011 y otros dispositivos periféricos.
\end{itemize}
Para este proyecto en particular, se optará por emplear el protocolo I2C debido a su compatibilidad con los componentes utilizados en la fuente disponibles en el mercado Argentino. La elección de este protocolo se fundamenta en su eficiencia y versatilidad, lo que lo hace idóneo para satisfacer los requisitos de comunicación de este sistema embebido.
La forma es que interactuan y se conectan los dispositivos entre si en base al protocolo es la siguiente:

\begin{figure}[H]
    \centering
    \includegraphics[scale=0.3]{./imagenes/i2cprotocol.jpg}
    \caption{Conexionado tipico de protocolo I2C.}
    \label{F:diagrama_protocolo_i2c}
\end{figure}

La forma en que tiene para comunicarse el arduino nano con los demás es mediante el canal I2C en donde toma el rol de único maestro en la comunicación. Ya que todos se encuentran conectados en una misma línea la forma de acceder a cada dispositivo independientemente con todos entre sí es mediante el uso de direcciones de 7 bits. En este caso estas serán las que se encuentran a continuación. Sin embargo se recuerda que estas en algunos casos son determinadas por la conexión del pin ADDRESS de los componentes, así que ante cualquier duda debe consultarse la hoja de datos correspondiente.
\begin{itemize}
    \item ADC ADS1115 ADDRESS: 0x48
    \item DAC MCP4725 ADDRESS: 0x60
    \item Display OLED ADDRESS: 0x3C
    \item Potenciómetro MCP4661 ADDRESS: ----
\end{itemize}

\subsection{Dependencias y Librerías Empleadas}

Una de las ventajas más destacadas de trabajar con Arduino es su activa y extensa comunidad, que ha desarrollado una vasta colección de librerías para simplificar la escritura de código y la implementación de funcionalidades avanzadas. Estas librerías permiten a los desarrolladores enfocarse en la lógica central de sus proyectos, sin tener que reinventar la rueda para tareas comunes. A continuación, se presentan las principales librerías utilizadas en este proyecto:

\begin{itemize}
    \item \textbf{Key.h}: Esta librería facilita la gestión de entradas de teclado, permitiendo la detección y el procesamiento eficiente de pulsaciones de teclas.
    \item \textbf{Keypad.h}: Utilizada para manejar teclados matriciales, esta librería simplifica la lectura de teclas y la interpretación de entradas de usuario.
    \item \textbf{Wire.h}: Esencial para la comunicación I2C, esta librería permite la interacción con una variedad de dispositivos periféricos compatibles con este protocolo, como sensores y expansores de E/S.
    \item \textbf{Adafruit\_ADS1X15.h}: Proporciona soporte para la familia de convertidores analógico-digital (ADC) ADS1X15 de Adafruit, permitiendo lecturas precisas de señales analógicas.
    \item \textbf{Adafruit\_GFX.h}: Una librería gráfica que proporciona primitivas de dibujo básicas, tales como líneas, círculos y texto, utilizada comúnmente en pantallas gráficas.
    \item \textbf{Adafruit\_SSD1306.h}: Especializada en el control de pantallas OLED basadas en el controlador SSD1306, esta librería facilita la visualización de información en pantallas compactas y de alta resolución.
    \item \textbf{Adafruit\_MCP4725.h}: Proporciona una interfaz sencilla para controlar el DAC MCP4725, permitiendo la generación de señales analógicas de manera precisa.
\end{itemize}

\section{Ensayos y simulación}
Para los ensayos de los modelos constructivos, se utilizó el software simulador de circuitos electrónicos Proteus 8 Professional. Este software proporciona una serie de herramientas que permiten evaluar el funcionamiento de los componentes implementados en el control digital de manera eficiente.
Una de las características más destacadas de Proteus 8 Professional, y la razón principal por la que se prefiere frente a otras alternativas, es su comunidad activa. Esta comunidad ha desarrollado librerías extensivas de componentes, incluidos microcontroladores Arduino. Estas librerías no solo incluyen las huellas (footprints) de los componentes, sino que también permiten programarlos de manera similar a como se haría con los dispositivos reales. Esta funcionalidad es particularmente valiosa, ya que permite al diseñador observar una simulación precisa de la interacción entre todos los elementos, sirviendo como base para la implementación con componentes físicos en etapas posteriores del desarrollo.
\begin{figure}[H]
    \centering
    \includegraphics[scale=0.4]{./imagenes/proteus_logo.jpg}
    \caption{Software Proteus 8 Professional.}
    \label{F:proteus_logo}
\end{figure}

\begin{figure}[H]
    \centering
    \includegraphics[scale=0.4]{./imagenes/proteus_esquema2.jpg}
    \caption{Esquemático de simulación en Proteus.}
    \label{F:esquematico_proteus}
\end{figure}

\subsection{Configuración del Ambiente de Trabajo}

Proteus 8 Professional incluye una amplia variedad de componentes preinstalados, que cubren la mayoría de los requisitos necesarios para las pruebas. Sin embargo, en el caso de los microcontroladores Arduino, estos deben descargarse e incorporarse manualmente en la carpeta de librerías del software.
Los dispositivos Arduino añadidos a Proteus mediante librerías permiten cargarles un código escrito en lenguaje C para su ejecución durante la simulación. Esto facilita la realización de ensayos y la verificación de las funcionalidades desarrolladas en el código.
El proceso para lograr esta integración consta de dos pasos. Primero, utilizando el Arduino IDE con el código deseado, se debe acceder a la sección de \textit{Sketch} y seleccionar \textit{Exportar binario compilado}. Esto generará una carpeta adicional junto al proyecto con extensión .ino, donde se crearán archivos con extensiones .hex y .elf, entre otros. Estos archivos deben seleccionarse al configurar las propiedades del componente Arduino en Proteus. Una vez completado este paso, el dispositivo funcionará como un Arduino real, permitiendo conectar los pines y suministrar la tensión según el diseño electrónico en el simulador para iniciar las pruebas.

\section{Resultados Experimentales}
Basado en el desarrollo descrito en la sección anterior, se obtuvo un modelo funcional de circuito y código que cumplía con los objetivos propuestos del sistema de control. Esto culminó en el ensayo físico de los componentes utilizando una base \textit{protoboard}, donde se comprobó que todos los elementos funcionaron según lo previsto. Así, se concluyó exitosamente el ensayo de esta sección, validando el diseño y su implementación práctica.
\begin{figure}[H]
    \centering
    \includegraphics[scale=0.08]{./imagenes/ensayo_digital.jpg}
    \caption{Esquemático de simulación en Proteus.}
    \label{F:esquematico_proteus}
\end{figure}

\subsection{Diagrama de Control del Software}

El software implementado en el microcontrolador para el control de la fuente de alimentación está diseñado siguiendo una estructura modular y organizada, lo que permite una adecuada secuenciación de las tareas y una toma de decisiones eficiente. El siguiente diagrama (Fig. X) muestra el flujo de trabajo del software, representando el orden en el que se ejecutan las distintas funciones y cómo se toman las decisiones dentro del microcontrolador para garantizar un control preciso de la salida.

El código del sistema se ha estructurado en cuatro secciones principales, cada una de las cuales cumple una función clave en la operación global:

\begin{itemize}
    \item \textbf{Manejo de Presión de Teclas:} Esta sección se encarga de detectar la pulsación de alguna de las teclas del panel de control por parte del usuario. En función de la tecla presionada, se ejecuta la acción correspondiente, como ajustar los valores de referencia de la tensión o corriente de salida, modificar configuraciones, o activar/desactivar la fuente. El correcto manejo de este módulo es esencial para la interacción eficiente del usuario con el sistema.
    \item \textbf{Manejo de la Interacción con el Encoder:} El encoder rotativo es un dispositivo crucial para la manipulación de las referencias en lazo de control. Esta sección del código capta los movimientos de rotación del encoder, ya sea en sentido horario o antihorario, y actualiza en tiempo real las referencias de tensión y corriente que el sistema debe alcanzar. Gracias a esta implementación, el usuario puede ajustar de manera precisa los valores de salida.
    \item \textbf{Algoritmo de Cálculo de la Acción de Control:} En esta sección se llevan a cabo los cálculos algorítmicos que determinan la acción de control óptima que se enviará al actuador. El algoritmo implementado aplica las ecuaciones correspondientes para calcular las señales de control de corriente y tensión, basándose en los errores detectados entre los valores de referencia y los valores actuales. La eficiencia en el cálculo de estas señales es crucial para garantizar una respuesta rápida y precisa del sistema.

    \item \textbf{Actualización del Display:} Esta sección maneja la comunicación con la pantalla de visualización, determinando el momento adecuado para su actualización y los datos que se deben mostrar. Se optimiza para que la pantalla se actualice con una frecuencia controlada, lo que permite al usuario monitorear el estado de la fuente de manera eficiente sin consumir recursos excesivos del sistema.
\end{itemize}

Además de estas secciones principales, se implementaron funciones complementarias, como la \textbf{Lectura de Datos} y la \textbf{Actualización de la Acción de Control}, las cuales se ejecutan periódicamente para mantener el sistema en funcionamiento estable. También se incluye una rutina de \textbf{Asignación de Pines}, que se activa al inicio del programa y determina configuraciones iniciales críticas, como la asignación de direcciones I2C de los dispositivos conectados, la configuración de los pines del relé, y otros parámetros estándares esenciales para la operación de la fuente.

\begin{figure}[H]
    \centering
    \includegraphics[scale=0.3]{./imagenes/DiagramaDeSoftware.jpg}
    \caption{Diagrama de procesos en el Arduino Nano.}
    \label{F:diagrama_de_procesos}
\end{figure}

\subsection{Detalles y Optimización del Código}
Para lograr un control eficiente y mantener un rendimiento óptimo durante la mayor cantidad de tiempo posible, se realizaron diversas optimizaciones en el código. Estas optimizaciones tienen como objetivo mejorar tanto la velocidad de respuesta del sistema como su estabilidad en condiciones de operación variables.
Una de las optimizaciones más relevantes es la decisión de actualizar el display una vez por segundo. Dado que la pantalla no es crítica para el control en tiempo real, esta frecuencia de actualización es suficiente para mantener al usuario informado sin sobrecargar los recursos del microcontrolador. Este ajuste reduce la carga en el sistema, permitiendo que el microcontrolador dedique más tiempo a las tareas críticas, como el cálculo de la acción de control.
Otro aspecto importante en la optimización del código fue la \textbf{Creación de Librerías}. Para mantener un código organizado, accesible y fácil de entender por cualquier usuario o desarrollador, se decidió estructurar el código en C++ mediante librerías dedicadas a cada una de las tareas principales. Esto no solo facilita el mantenimiento del código, sino que también mejora su modularidad, permitiendo la reutilización de secciones de código en futuros proyectos o en modificaciones del sistema. Las librerías contienen todos los parámetros relevantes para cada tarea, lo que garantiza una clara separación de responsabilidades y una estructura de código eficiente y escalable.
Estas optimizaciones en conjunto contribuyen significativamente al rendimiento general del sistema, asegurando que el control sea robusto y que el software sea fácil de mantener y extender en el futuro.
\chapter{Modelado y construcción del PCB.}
% ----------------------

\label{C:Modelado y construcción del PCB.}

\section{Software y herramientas de diseño empleadas.}

A partir de los circuitos desarrollados en los capítulos anteriores, se procedió a modelar una placa de circuito impreso (PCB) personalizada. Esta placa está diseñada para integrar todos los componentes necesarios y crear un prototipo funcional que permita realizar ensayos sobre materiales en una superficie comprimida.
La modelación y diseño del PCB se llevaron a cabo utilizando el software KiCad, reconocido por su amplia gama de herramientas de personalización de componentes. Este software permite a los diseñadores lograr un alto nivel de precisión y calidad en sus diseños, adecuándose a las habilidades específicas de cada usuario.
El proceso de diseño incluyó la disposición estratégica de los componentes para optimizar el rendimiento del circuito, así como la consideración de factores como la disipación de calor, la integridad de la señal y la minimización de interferencias electromagnéticas. Además, se realizaron varias iteraciones del diseño para asegurar que el PCB final cumpliera con todos los requisitos técnicos y de funcionamiento necesarios para los ensayos planificados.
El uso de KiCad facilitó la creación de un diseño detallado y eficiente, permitiendo visualizar en todo momento el aspecto final del PCB y realizar ajustes necesarios antes de proceder a su fabricación.
\begin{figure}[H]
    \centering
    \includegraphics[scale=0.03]{./imagenes/KiCad-Logo.svg_.png}
    \caption{Logo del software Kicad.}
    \label{F:kicad}
\end{figure}

\section{Construcción del primer prototipo.}
La fase de construcción se inició con el desmontaje de la placa analógica de la fuente utilizada en un proyecto anterior, la cual se caracterizaba por sus atributos de control predominantemente analógicos. Este proceso permitió la recuperación de una variedad de materiales que, en su mayoría, se emplearían en el desarrollo del nuevo prototipo de fuente digital. Entre los componentes rescatados se encuentran resistencias, capacitores, disipadores de calor y borneras, entre otros. La reutilización de estos elementos fue posible gracias a la topología de la nueva fuente digital, que permitía su integración sin comprometer el diseño ni la funcionalidad del prototipo. 
El proceso de desmontaje y reutilización de componentes se llevó a cabo meticulosamente, asegurando que cada pieza recuperada estuviera en condiciones óptimas para su apropiada colocación en la nueva placa. Este esfuerzo contribuyó a la eficiencia del proyecto y a la racionalización de recursos, destacando la importancia de la sostenibilidad y la economía circular en el ámbito del diseño y construcción de dispositivos electrónicos.

\begin{figure}[H]
    \centering
    \includegraphics[scale=0.1]{./imagenes/fotos/desmontada.jpg}
    \caption{Después del desmontaje de la placa.}
    \label{F:desmontaje_de_la_placa}
\end{figure}

\begin{figure}[H]
    \centering
    \includegraphics[scale=0.1]{./imagenes/fotos/placa_original.jpg}
    \caption{Antes desmontaje de la placa.}
    \label{F:placa_original}
\end{figure}

A continuación, se presenta el diseño del prototipo utilizado, el cual incorpora todos los elementos necesarios para la realización de las pruebas de funcionamiento. La característica principal de este PCB es su capacidad para integrar en un espacio compacto de 15x20 cm todos los componentes que anteriormente estaban dispersos en el modelo anterior.
Una excepción notable en el diseño es la ubicación de la pantalla y el teclado, que se ha decidido mantener separados del PCB principal. Esta decisión se tomó debido a que no tendría sentido práctico incluir estos elementos directamente sobre la placa. En su lugar, se emplearon pines de salida, como borneras, para conectar estos componentes externos, facilitando su integración y operación.
El diseño resultante, que se muestra en la imagen adjunta, incluye también una representación tentativa en 3D del PCB. En esta representación se pueden observar las disposiciones de los componentes y la estructura general del prototipo. Es importante destacar que, para evitar daños y facilitar el acceso y reemplazo, algunos de los componentes están montados sobre tiras de pines hembra en lugar de estar soldados directamente sobre la placa. Esta configuración no solo mejora la durabilidad y preservación del prototipo, sino que también permite una mayor flexibilidad en la realización de pruebas y modificaciones. La inclusión de un modelo 3D en el diseño ayuda a visualizar la disposición y la accesibilidad de los componentes, asegurando que el montaje y el mantenimiento del PCB sean lo más eficientes posible.
\begin{figure}[H]
    \centering
    \includegraphics[scale=0.5]{./imagenes/pcb_v1.jpg}
    \caption{Primer prototipo de PCB.}
    \label{F:PCB_V1}
\end{figure}
\begin{figure}[H]
    \centering
    \includegraphics[scale=0.5]{./imagenes/prototipo1.jpg}
    \caption{Vista 3D del primer prototipo.}
    \label{F:PCB_3D}
\end{figure}
\begin{figure}[H]
    \centering
    \includegraphics[scale=0.1]{./imagenes/fotos/montaje.jpg}
    \caption{Montaje de los componentes en la placa.}
    \label{F:montaje_componentes}
\end{figure}

\section{Ensayo de laboratorio y pruebas prácticas.}
Una vez verificada la continuidad de las pistas, el adecuado funcionamiento de los componentes, y los niveles de tensión en varios puntos clave, se procedió a energizar la fuente con todos los transformadores, tomando todas las precauciones necesarias para evitar daños a los componentes.
A partir de este punto, se realizó una serie de pruebas y ajustes detallados para garantizar el correcto funcionamiento de la fuente. Estas pruebas incluyen la verificación de la respuesta del sistema bajo diversas condiciones de carga y la evaluación de la estabilidad del lazo de control. El uso del osciloscopio fue fundamental en este proceso, ya que permitió observar en tiempo real cómo el lazo de control afectaba la salida de la fuente, aspecto crucial para el correcto desempeño del dispositivo.
Durante estas pruebas, se monitorizaron diversos parámetros, tales como la tensión de salida, la respuesta transitoria, y el comportamiento ante variaciones en la carga. Cada ajuste se realizó con el objetivo de optimizar el rendimiento del prototipo, asegurando que este cumpliera con los requisitos especificados y operará de manera eficiente y estable.
El resultado de estos ensayos fue satisfactorio, evidenciando que el diseño y la construcción del PCB fueron exitosos pero sin embargo no del todo concluyentes. Por de manera de buscar la excelencia en el producto final se obtuvieron las siguientes observaciones que serán tomadas en cuenta para el siguiente prototipo.  
\begin{figure}[H]
    \centering
    \includegraphics[scale=0.1]{./imagenes/fotos/osciloscopio.jpg}
    \caption{Ensayo con osciloscopio de la placa.}
    \label{F:esayos_y_pruebas}
\end{figure}

\section*{Observaciones}
\begin{itemize}
    \item Los transistores comienzan a operar con una acción de control mínima de 0.3V.
    \item Al desconectarse una carga el capacitor de salida incrementa su voltaje en cuestión de unos milisegundos debido a que la acción de control no disminuye de forma tan instantánea. Por lo que luego de esto, es necesario un tiempo extra para que el capacitor se descargue y se estabilicen nuevamente los niveles de tensión. Que además, como no tiene carga conectada deberá ser más lenta la descarga del capacitor.
    \item Cuando el ADC no tiene ningún valor como referencia proveniente desde el Arduino este pone su voltaje de salida en 2.5V.
\end{itemize}

\section*{Mejoras a realizar}
\begin{itemize}
    \item Ajuste de constantes de controlador.
    \item Ajuste de frecuencia de muestreo.
    \item Mejora de la estrategia de control.
\end{itemize}


\input{contenido/9_Ensayos_y_pruebas.tex}
\input{contenido/Instrucciones_uso.tex}


%\input{contenido/4_clasificacion_y_def.tex}

% 9. BILIOGRAFÍA
\printbibliography[title={Bibliografía}]

% 8. APÉNDICES
\appendix
% si necesitamos agregar algún apéndice activamos esto
\chapter{Apéndice de ejemplo}
\label{C:anexo-comunicación-serial}

\lipsum[2-4]



\backmatter

%%%%%%%%%%%%%%%%%%%
\end{document}
%%%%%%%%%%%%%%%%%%%
